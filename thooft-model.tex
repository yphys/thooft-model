\documentclass{article}
\usepackage[utf8]{inputenc}
\usepackage[margin=1.5in]{geometry}
\usepackage{amsmath}
\usepackage{amssymb}
\usepackage{xcolor}
\usepackage{hyperref}
\usepackage{slashed}

\hypersetup{
        bookmarksnumbered,
        unicode,                        % use with \texorpdfstring
        colorlinks,                     % avoid stupid boxes
        citecolor=[rgb]{.9,0,.5},       % \cite
        urlcolor=[rgb]{0,0,1},  % \href
        linkcolor=[rgb]{0,.7,0} % \ref , toc
        }

\usepackage[
    %backend=biber, 
    natbib=true,
    style=numeric,
    sorting=none
]{biblatex}

\newcommand{\bref}[1]{(\ref{#1})}
\newcommand{\pd}{\partial}
\title{Notes on `t Hooft's model}
\author{Yu-Ping Wang}
\date{\today}

\begin{document}
\maketitle
\tableofcontents


\section{2D QCD}
\paragraph{}The Lagrangian of two-dimensional QCD with $U(N)$ gauge group with $m$-flavors of Dirac fermions.
\[
{\cal L} = \frac{1}{4}\textrm{tr}F_{\mu\nu}F^{\mu\nu} - q_{a}\left(\slashed{D}-m^2\right)\overline{q}^{a}, \quad a = 1, \cdots m. 
\]
We shall set up some basic notations.  $\slashed{D} = \gamma^{\nu}D_{\nu}$ is the Dirac derivatives, and 
\[
F_{\mu\nu} = \pd_{\mu}A_{\nu}-\pd_{\nu}A_{\mu} + g \left[A_{\mu}, A_{\nu}\right], \quad  D_{\mu}q^{a} = \pd_{\mu}q^{a} + g \overline{A}_{\mu}\cdot q^{a}.
\]
$\overline{A}_{\mu} = A_{\mu} - \frac{1}{N}I \textrm{tr}A_{\mu}$ is the traceless part of $A_{\mu}$.

Its more convenient to use the light-cone coordinate.
\[
x_{\mu} = (x_{+}, x_{-}), \quad x_{\pm}  = \frac{1}{\sqrt{2}}\left( x_{0}\pm x_{1}\right).
\]
The momentum in light-cone coordinate is denote similarly $p_{\pm}=\frac{1}{\sqrt{2}}\left( p_{0}\pm p_{1}\right)$, and the inner product is $a_{\mu}b^{\mu} = a_{+}b_{-} + a_{-}b_{+}$. This means that when raising/lowering the indices, the plus and minus sing needs to be exchanged.

The gauge transformation of $A_{\mu}$ is $\delta A_{\mu} = $

\nocite{*}
\printbibliography

\end{document}
