\documentclass{article}
\usepackage[utf8]{inputenc}
\usepackage[margin=1.5in]{geometry}
\usepackage{amsmath}
\usepackage{amssymb}
\usepackage{xcolor}
\usepackage{hyperref}
\usepackage{slashed}
\usepackage{subcaption}
\usepackage{tikz}
\usepackage{tikz-feynman}
\usepackage{lipsum}
\usepackage{array}
\hypersetup{
        bookmarksnumbered,
        unicode,                        % use with \texorpdfstring
        colorlinks,                     % avoid stupid boxes
        citecolor=[rgb]{.9,0,.5},       % \cite
        urlcolor=[rgb]{0,0,1},  % \href
        linkcolor=[rgb]{0,.7,0} % \ref , toc
        }

\usepackage[
    %backend=biber, 
    natbib=true,
    style=numeric,
    sorting=none
]{biblatex}

\usepackage{feynmp-auto}
%\usepackage{feynman}

\newcommand{\bref}[1]{(\ref{#1})}
\newcommand{\pd}{\partial}
\title{Notes on `t Hooft's model}
\author{Yu-Ping Wang}
\date{\today}

\begin{document}
\maketitle
\tableofcontents


\section{2D QCD}
\paragraph{}The Lagrangian of two-dimensional QCD with $U(N)$ gauge group with $m$-flavors of Dirac fermions.
\[
{\cal L} = \frac{1}{4}\textrm{tr}F_{\mu\nu}F^{\mu\nu} - q_{a}\left(i\slashed{D}-m_{a}\right)\overline{q}^{a}, \quad a = 1, \cdots m. 
\]
We shall set up some basic notations.  $\slashed{D} = \gamma^{\nu}D_{\nu}$ is the Dirac derivatives, and 
\[
F_{\mu\nu} = \pd_{\mu}A_{\nu}-\pd_{\nu}A_{\mu} + g \left[A_{\mu}, A_{\nu}\right], \quad  D_{\mu}q^{a} = \pd_{\mu}q^{a} + g \overline{A}_{\mu}\cdot q^{a}.
\]
$\overline{A}_{\mu} = A_{\mu} - \frac{1}{N}I \textrm{tr}A_{\mu}$ is the traceless part of $A_{\mu}$.

Its more convenient to use the light-cone coordinate.
\[
x_{\mu} = (x_{+}, x_{-}), \quad x_{\pm}  = \frac{1}{\sqrt{2}}\left( x_{0}\pm x_{1}\right).
\]
The momentum in light-cone coordinate is denote similarly $p_{\pm}=\frac{1}{\sqrt{2}}\left( p_{0}\pm p_{1}\right)$, and the inner product is $a_{\mu}b^{\mu} = a_{+}b_{-} + a_{-}b_{+}$. This means that when raising/lowering the indices, the plus and minus sing needs to be exchanged.

In two-dimensons, we are allowed to choose a gauge of $A_{\mu}$ such that $A_{-} = A^{+}$ =0. We also list out the explicit values of the gamma matrices $\gamma_{\mu}$.

\[
\gamma_{\mu} \rightarrow \gamma_{\pm}, \quad \gamma_{+} = 
\begin{pmatrix}
 0 & 1\\
 0 & 0   
\end{pmatrix},\;
\gamma_{-} = 
\begin{pmatrix}
 0 & 0\\
 1 & 0   
\end{pmatrix}.
\]
In this gauge, the Lagrangian could be explicit written down
\[
{\cal L} = -\frac{1}{2}\textrm{tr}\left(\pd_{-}A_{+}\right)^2 - q_{a}\left(i\gamma_{+}\pd_{-}+ i\gamma_{-}\pd_{+} + m_{a} +g\gamma_{-}\overline{A}_{+} \right)\overline{q}^{a}.
\]

The advantage of choosing the light-cone gauge is that there are no direct interactions between the gluons. Making the Feynman diagrams much simpler.  Since $A_{\mu}$ is a $U(N)$ matrix, we could use the double line notation for the gluon propagators. 

The Feynman rules are
\begin{center}
\begin{tabular}{m{0.3\textwidth} m{0.3\textwidth} m{0.3\textwidth}}
\begin{tikzpicture}[baseline={([yshift=-1.5ex]i1.base)}]
        \begin{feynman}
        \vertex at (0, 1) (i1) ;
        \vertex at (0, 0.8) (i2) ;
        \vertex at (3 , 1) (i3);
        \vertex at (3, 0.8) (i4);
        \vertex at (1.5, 0.8) (a);
        \diagram*{
            (i1) -- [fermion] (i3),
            (i4) -- [fermion] (i2),
        };
    \end{feynman}
\end{tikzpicture} &\( -\frac{i}{k_{-}^2}\) & Gluon propagator\\[1em]
\begin{tikzpicture} [baseline={([yshift=-1.5ex]i1.base)}]
    \begin{feynman}
        \vertex at (0, 1) (i1) {\(i\)} ;
        \vertex at (3 , 1) (i3){\(j\)} ;
        \vertex at (1.5, 1) (a);
        \vertex at (1.5, 1.3)(b) {\(a\)};
        \diagram*{
            (i1) -- [fermion] (i3),
        };
    \end{feynman}
\end{tikzpicture} & \( \frac{i}{\gamma_{+}p_{-} + \gamma_{-}p_{+} -m_{a} +i\epsilon}\) &\\ [1em]
& = \(\frac{\gamma_{+}p_{-} + \gamma_{-}p_{+} + m_{a}}{2p_{+}p_{-} - m_{a}^2 + i\epsilon} \)& Fermion propagator\\[1em]
    \begin{tikzpicture}[baseline={([yshift=-5.5ex]i1.base)}]
        \begin{feynman}
        \vertex at (0, 0) (i1) ;
        \vertex at (0, 0.2) (i2) ;
        \vertex at (1.7 , 0) (a);
        \vertex at (1.7, 0.2) (b);
        \vertex at (2.5, 1.2) (i3) {\(i\)};
        \vertex at (2.5, -1) (i4)  {\(j\)};
        \diagram*{
            (i4) -- [fermion] (a) -- [fermion] (i1),
            (i2) -- [fermion] (b) -- [fermion] (i3),
        };
    \end{feynman}
\end{tikzpicture}
    & \(  ig \gamma_{-} \)& Gluon-quark interaction\\[3em]
\begin{tikzpicture}[baseline={([yshift=-5.5ex]i1.base)}]
        \begin{feynman}
        \vertex at (0, 0) (i1) ;
        \vertex at (0, 0.2) (i2) ;
        \vertex at (2.5, 1.2) (i3) {\(i\)};
        \vertex at (2.5, -1) (i4) {\(j\)};
        \vertex at (1.7 , 0)  (a);
        \vertex at (1.7, 0.2) (b);
        \vertex at (1.5 , 0)  (a1);
        \vertex at (1.5, 0.2) (b1);
        \diagram*{
            (i4) -- [fermion] (a) -- (b) -- [fermion] (i3),
            (i2) -- [fermion] (b1) --  (a1) -- [fermion] (i1),
        };
    \end{feynman}
\end{tikzpicture}
& \(-\frac{i}{N}g\gamma_{-}\) & Higher order gluon-quark interaction.
\end{tabular}
\end{center}  

\nocite{*}
\printbibliography

\end{document}
